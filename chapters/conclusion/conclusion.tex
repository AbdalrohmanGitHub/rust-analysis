% !TEX root =  master.tex
\chapter{Zusammenfassung}

\nocite{*}

Dieses Kapitel enthält die Zusammenfassung der Arbeit mit Fazit und Ausblick.

\section{Fazit}

In dieser Arbeit wurde zunächst ein Überblick zu den unterschiedlichen Kernkonzepten und Besonderheiten der Programmiersprache Rust wie Ownership, Borrowing, Lifetimes und unsicheren Code, der in sicheren APIs gekapselt ist, gegeben. Rust ist die erste von der Industrie unterstützte Sprache, die den jahrzehntelangen Kompromiss im Programmiersprachenentwurf zwischen Sicherheit und Kontrolle überwinden konnte. Sie ist sowohl eine typsichere Sprache als auch eine Systemprogrammiersprache. Hierbei unterscheidet sich Rust von den anderen Sprachen, die die gleiche Garantie für Typsicherheit bieten, darin, dass sie dies ohne Garbage Collector oder manuelle Speicherverwaltung tut. Rust ist nicht wirklich eine objektorientierte Sprache, obwohl sie einige objektorientierte Merkmale aufweist. Rust ist keine funktionale Sprache, obwohl sie dazu neigt, die Einflüsse auf das Ergebnis einer Berechnung deutlicher zu machen, wie es funktionale Sprachen tun. Rust ist für Entwickler und Projekte gedacht, bei denen nicht nur Leistung und low-level-Optimierungen wichtig sind, sondern bei denen auch eine sichere und stabile Ausführungsumgebung benötigt wird. Rust fügt eine Menge funktionaler High-Level-Programmiertechniken innerhalb der Sprache hinzu, so dass sie sich gleichzeitig wie eine Low-Level- und eine High-Level-Sprache anfühlt. Rust ähnelt C und \texttt{C++} bis zu einem gewissen Grad, aber viele Idiome aus diesen Sprachen treffen eher nicht zu, so dass der typische Rust-Code C oder \texttt{C++}-Code nicht sehr ähnlich ist.

Rust ist für die Implementierung jener fundamentalen Systemschichten konzipiert, die eine starke Leistung und eine umfassende Kontrolle über die Ressourcen erfordern, aber dennoch das grundlegende Niveau der Vorhersagbarkeit garantieren, das die Typsicherheit bietet. Dadurch wird Entwicklern möglich, die Kosten ihres Codes vorherzusehen.
Rust hält den Ressourcenverbrauch niedrig, ohne hierbei, die Speichersicherheit zu beeinträchtigen. Rust bringt moderne Werkzeuge in den Bereich der Systemprogrammierung. Zwar gibt es immer noch Best Practices und Werkzeuge, die verbessert werden könnten, aber der Erfolg von Rust ist auf langfristigen Erfolg ausgerichtet.


